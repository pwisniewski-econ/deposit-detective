% Options for packages loaded elsewhere
\PassOptionsToPackage{unicode}{hyperref}
\PassOptionsToPackage{hyphens}{url}
\PassOptionsToPackage{dvipsnames,svgnames,x11names}{xcolor}
%
\documentclass[
  12pt,
]{article}

\usepackage{amsmath,amssymb}
\usepackage{iftex}
\ifPDFTeX
  \usepackage[T1]{fontenc}
  \usepackage[utf8]{inputenc}
  \usepackage{textcomp} % provide euro and other symbols
\else % if luatex or xetex
  \usepackage{unicode-math}
  \defaultfontfeatures{Scale=MatchLowercase}
  \defaultfontfeatures[\rmfamily]{Ligatures=TeX,Scale=1}
\fi
\usepackage{lmodern}
\ifPDFTeX\else  
    % xetex/luatex font selection
\fi
% Use upquote if available, for straight quotes in verbatim environments
\IfFileExists{upquote.sty}{\usepackage{upquote}}{}
\IfFileExists{microtype.sty}{% use microtype if available
  \usepackage[]{microtype}
  \UseMicrotypeSet[protrusion]{basicmath} % disable protrusion for tt fonts
}{}
\makeatletter
\@ifundefined{KOMAClassName}{% if non-KOMA class
  \IfFileExists{parskip.sty}{%
    \usepackage{parskip}
  }{% else
    \setlength{\parindent}{0pt}
    \setlength{\parskip}{6pt plus 2pt minus 1pt}}
}{% if KOMA class
  \KOMAoptions{parskip=half}}
\makeatother
\usepackage{xcolor}
\setlength{\emergencystretch}{3em} % prevent overfull lines
\setcounter{secnumdepth}{5}


\providecommand{\tightlist}{%
  \setlength{\itemsep}{0pt}\setlength{\parskip}{0pt}}\usepackage{longtable,booktabs,array}
\usepackage{calc} % for calculating minipage widths
% Correct order of tables after \paragraph or \subparagraph
\usepackage{etoolbox}
\makeatletter
\patchcmd\longtable{\par}{\if@noskipsec\mbox{}\fi\par}{}{}
\makeatother
% Allow footnotes in longtable head/foot
\IfFileExists{footnotehyper.sty}{\usepackage{footnotehyper}}{\usepackage{footnote}}
\makesavenoteenv{longtable}
\usepackage{graphicx}
\makeatletter
\def\maxwidth{\ifdim\Gin@nat@width>\linewidth\linewidth\else\Gin@nat@width\fi}
\def\maxheight{\ifdim\Gin@nat@height>\textheight\textheight\else\Gin@nat@height\fi}
\makeatother
% Scale images if necessary, so that they will not overflow the page
% margins by default, and it is still possible to overwrite the defaults
% using explicit options in \includegraphics[width, height, ...]{}
\setkeys{Gin}{width=\maxwidth,height=\maxheight,keepaspectratio}
% Set default figure placement to htbp
\makeatletter
\def\fps@figure{htbp}
\makeatother

\usepackage{graphicx}
\usepackage{pdflscape}
\usepackage{pdfpages}
\newcommand*{\boldone}{\text{\usefont{U}{bbold}{m}{n}1}}
\usepackage[a4paper, portrait, footnotesep=0.75cm, margin=2.54cm]{geometry}
\usepackage{enumitem}
\usepackage{parskip}
\usepackage{titling}
\linespread{1.5}
\usepackage[T1]{fontenc}
\usepackage[hidelinks]{hyperref}
\hypersetup{linkcolor={black}}
\usepackage{amsmath}
\usepackage{amsfonts}
\usepackage[normalem]{ulem}
\usepackage{times}
\usepackage{sectsty}
\usepackage{ressources/tikz/tikzit}
\input{ressources/tikz/tikz.tikzstyles}
\newcommand{\ts}{\textsuperscript}
\makeatletter
\@ifpackageloaded{caption}{}{\usepackage{caption}}
\AtBeginDocument{%
\ifdefined\contentsname
  \renewcommand*\contentsname{Table of contents}
\else
  \newcommand\contentsname{Table of contents}
\fi
\ifdefined\listfigurename
  \renewcommand*\listfigurename{List of Figures}
\else
  \newcommand\listfigurename{List of Figures}
\fi
\ifdefined\listtablename
  \renewcommand*\listtablename{List of Tables}
\else
  \newcommand\listtablename{List of Tables}
\fi
\ifdefined\figurename
  \renewcommand*\figurename{Figure}
\else
  \newcommand\figurename{Figure}
\fi
\ifdefined\tablename
  \renewcommand*\tablename{Table}
\else
  \newcommand\tablename{Table}
\fi
}
\@ifpackageloaded{float}{}{\usepackage{float}}
\floatstyle{ruled}
\@ifundefined{c@chapter}{\newfloat{codelisting}{h}{lop}}{\newfloat{codelisting}{h}{lop}[chapter]}
\floatname{codelisting}{Listing}
\newcommand*\listoflistings{\listof{codelisting}{List of Listings}}
\makeatother
\makeatletter
\makeatother
\makeatletter
\@ifpackageloaded{caption}{}{\usepackage{caption}}
\@ifpackageloaded{subcaption}{}{\usepackage{subcaption}}
\makeatother
\makeatletter
\@ifpackageloaded{tcolorbox}{}{\usepackage[skins,breakable]{tcolorbox}}
\makeatother
\makeatletter
\@ifundefined{shadecolor}{\definecolor{shadecolor}{HTML}{5b5b5b}}{}
\makeatother
\makeatletter
\@ifundefined{codebgcolor}{\definecolor{codebgcolor}{HTML}{fafafa}}{}
\makeatother
\makeatletter
\ifdefined\Shaded\renewenvironment{Shaded}{\begin{tcolorbox}[enhanced, sharp corners, borderline west={3pt}{0pt}{shadecolor}, frame hidden, boxrule=0pt, breakable, colback={codebgcolor}]}{\end{tcolorbox}}\fi
\makeatother

\ifLuaTeX
  \usepackage{selnolig}  % disable illegal ligatures
\fi
\usepackage{bookmark}

\IfFileExists{xurl.sty}{\usepackage{xurl}}{} % add URL line breaks if available
\urlstyle{same} % disable monospaced font for URLs
\hypersetup{
  colorlinks=true,
  linkcolor={blue},
  filecolor={Maroon},
  citecolor={black},
  urlcolor={gray},
  pdfcreator={LaTeX via pandoc}}


\author{}
\date{}

\begin{document}

\pagenumbering{gobble}
\includepdf{ressources/title-page/title-page.pdf}
\newpage
\pagenumbering{arabic}
\setcounter{page}{1}



\clearpage

\section{Introduction and Relevant
Literature}\label{introduction-and-relevant-literature}

\newpage

\section{Data}\label{data}

\subsection{Marketing Campaign}\label{marketing-campaign}

\begin{figure}
\centering
\caption{Campaign Process and Observed Data}
\tikzfig{ressources/tikz/fig1}
\end{figure}

\subsection{Descriptive Statistics}\label{descriptive-statistics}

\subsection{Exploratory Analysis}\label{exploratory-analysis}

\newpage

\section{Methodology and Objectives}\label{methodology-and-objectives}

\newpage

\section{Weekdays}\label{weekdays}

\subsection{PICO process}\label{pico-process}

\subsection{DAG and Hypothesis}\label{dag-and-hypothesis}

\subsection{Results}\label{results}

\subsection{Heterogeneity Analysis}\label{heterogeneity-analysis}

\newpage

\section{Time of the month}\label{time-of-the-month}

\subsection{PICO process}\label{pico-process-1}

\subsection{DAG and Hypothesis}\label{dag-and-hypothesis-1}

\subsection{Results}\label{results-1}

\subsection{Heterogeneity Analysis}\label{heterogeneity-analysis-1}

\newpage

\section{Macro Context}\label{macro-context}

This section analyzes macroeconomic determinants of the marketing
campaign, focusing on the impact of interest rates on product take-up
rates. The specific rate offered to consumers is unobserved; however,
the product, a term deposit, is among the fastest to react to changes in
market conditions. The 12-month EURIBOR rate is used as a proxy, given
its strong predictive power for such products. The objective is to
estimate consumer demand elasticity with respect to interest rates. This
has two key implications for the offering bank: it informs the expected
effect of offering a more competitive rate or supports assessment of
campaign viability when pricing power is limited and rates are
determined by market conditions.

\subsection{PICO summary}\label{pico-summary}

\textbf{Population}: The population of interest consists of bank
consumers; however, due to previously mentioned data limitations, the
analysis is restricted to clients who participated in the marketing
campaign. To simplify the study design, only clients who received a
single call are included.

\textbf{Intervention}: Variation in market interest rates which was
particularly pronounced during the 2008--2010 period is leveraged to
estimate demand elasticity.

\textbf{Comparison}: Clients exposed to lower market interest rates
serve as a control group for those exposed to higher interest rates.

\textbf{Outcome}: The primary outcome of interest is the take-up of the
offered savings product, measured as a change in the take-up rate.

\subsection{DAG and Hypothesis}\label{dag-and-hypothesis-2}

The main challenge in identifying the effect of interest rates on
take-up rates is endogeneity. Specifically, interest rates represent
market prices for capital and are thus the joint outcome of both supply
and demand.

This is problematic because interest rates are correlated with demand
shocks, which directly influence take-up behavior. A clear example is
the bankruptcy of Lehman Brothers, which marked the onset of the great
financial crisis. This significantly altered investor risk preferences,
leading to a substantial inflow of capital into safer financial
instruments such as deposits or fixed term accounts.

The directed acyclic graph below summarizes the primary causal
relationships assumed in this section.

\begin{figure}
\centering
\caption{Interest Rates and Take-Up}
\vspace{0.25cm}
\tikzfig{ressources/tikz/fig2}
\end{figure}
\vspace{-1cm}

Under the structure outlined in the DAG, the lagged interest rate at
period \(t-1\) can be considered a valid instrument, provided that
variables influenced by this rate are appropriately controlled for. The
intuition behind this instrument is based on interest rate stickiness:
financial products are often priced based on previously observed rates,
while current market rates reflect present expectations rather than past
ones. Therefore, contemporaneous demand shocks can be assumed to be
uncorrelated with lagged interest rates.

This type of instrument and identification strategy is well-established
in the macroeconomic literature. However, the exogeneity of the
instrument cannot be formally tested. The absence of strong alternative
instruments in our dataset rule out the possibility of conducting
over-identification tests such as the Sargan--J test.

\subsection{Results}\label{results-2}

To estimate causal effects, three methodological approaches are employed
and compared: (i) a standard two-stage least squares (2SLS) estimator
applied to a linear probability model, (ii) a double machine learning
(DML) estimator for a partially linear instrumental variables (IV)
regression, and (iii) a probit IV regression.

\newpage

TABLE HERE DML and Probit are close 2SLS very different

{[}Discussion of 2SLS + equations{]}

{[}Discussion of DML + equations{]}

{[}Discussion of Probit + equations{]}

{[}Conclusion on results{]}

\newpage

\section{Conclusion}\label{conclusion}

\subsection{Discussion}\label{discussion}

\subsection{Limitations}\label{limitations}

\newpage

\section{Bibliography}\label{bibliography}

\newpage

\setcounter{section}{0}
\renewcommand\thesection{\Alph{section}}

\section{Appendix}\label{appendix}

\subsection{Robustness}\label{robustness}

\subsection{Robustness}\label{robustness-1}

\subsection{Robustness}\label{robustness-2}




\end{document}
